\documentclass[a4paper]{article}
\usepackage{fontspec}
\usepackage{fontenc}
\usepackage{extarrows}
\usepackage{chemfig}
\usepackage[version=4]{mhchem}
\usepackage{amsmath}
\usepackage{amssymb}
\usepackage{siunitx}
\usepackage{bigfoot}
\usepackage{fancyvrb}
\usepackage{expl3}
\usepackage{calc}
\usepackage{geometry}
\geometry{left=2.5cm,right=2.5cm,top=2cm,bottom=3cm}
\setmainfont{Hiragino Sans GB}

\renewcommand\contentsname{目录}

\title{元素化学笔记整理}
\author{胡译文}
\date{\today}

\makeatletter
\newcommand{\figcaption}{\def@captype{figure}\caption}
\newcommand{\tabcaption}{\def@captype{table}\caption}
\makeatother


\begin{document}
	\maketitle
	\renewcommand\contentsname{目录}
	\tableofcontents
	\newpage
	
	
	\section{\ce{Na}}
	
	\subsection{\ce{Na}单质}
	
	\subsubsection{物理性质}
	\begin{itemize}
		\item 银白色固体,有金属性光泽;
		\item 密度介于水和煤油之间,用煤油或石蜡保存;
		\item 熔点低;
		\item 质地较软,可以用小刀切割。
	\end{itemize}
	
	\subsubsection{化学性质}
		\paragraph{与非金属单质反应} 
			\begin{itemize}
				\item $\left\{\begin{array}{lr}
						\ce{4Na + O2 -> 2Na2O}\\
						\ce{2Na + O2 ->[\Delta] Na2O2}\\
					\end{array}\right.$
				\item $\ce{2Na + S -> Na2S}$
				\item $\ce{2Na + H2 ->[\Delta] 2NaH}$
				\item $\left\{\begin{array}{lr}
						\ce{2Na + Br2 -> 2NaBr}\\
						\ce{2Na + Cl2 ->[{点燃}] 2NaCl}\\
					\end{array}\right.$
			\end{itemize}
			\paragraph{与水反应}
			$\ce{2Na + 2H2O -> 2NaOH + H2 ^}$
			\begin{itemize}
				\item 浮:钠的密度比水小
				\item 溶:反应放热,钠的熔点低
				\item 游:生成氢气推动钠
				\item 响:反应剧烈
				\item 红:生成\ce{NaOH}遇到酚酞变红
			\end{itemize}
			\paragraph{与盐酸反应}
			$\ce{2Na + 2HCl -> 2NaCl + H2 ^}$
			\paragraph{与碱反应}
			实质是先与水反应,产物再和盐反应。
			\paragraph{与盐溶液反应}
			实质是先与水反应,产物再和盐反应(钠不能与盐溶液发生置换反应)。
			\begin{itemize}
				\item 钠与硫酸铜溶液
				$\left\{\begin{array}{lr}
					\ce{2Na + 2H20 -> 2NaOH + H2 ^}\\
					\ce{2NaOH + CuSO4 -> Na2SO4 + Cu(OH)2 v}\\
				\end{array}\right.$
			\end{itemize}
			\paragraph{与\ce{CO2}反应}
			$\left\{\begin{array}{lr}
				\ce{4Na + CO2 ->[\Delta] 2Na2O + C}\\
				\ce{4Na + 3CO2 ->[\Delta] 2Na2CO3 + C}\\
			\end{array}\right.$
		
	\subsubsection{钠的制取}
	$\left\{\begin{array}{lr}
		\ce{2NaCl(l) ->[{电解}] 2Na + Cl2 ^}\\
		\ce{2NaOH(l) ->[{电解}] 2Na + O2 ^ + H2 ^}\\
	\end{array}\right.$
	
	\subsubsection{钠的用途}
	\begin{itemize}
		\item 冶炼金属:$\ce{4Na + TiCl4(l) -> 4NaCl + Ti}$
		\item 原子反应导热剂
		\item 钠光灯
	\end{itemize}
	
	
	\subsection{\ce{Na}的化合物}
	
	\subsubsection{氧化钠和过氧化钠}
	\paragraph{比较氧化钠和过氧化钠}
	\renewcommand\arraystretch{2}
	\begin{center}
	\begin{tabular}{|c|c|c|}
		\hline
		名称&氧化钠&过氧化钠\\\hline
		化学式&\ce{Na2O}&\ce{Na2O2}\\\hline
		物理性质&白色固体&淡黄色固体\\\hline
		氧化物类型&碱性氧化物&过氧化物\\\hline
		制取&$\ce{4Na + O2 -> 2NaO}$&$\ce{2Na + O2 ->[\Delta] Na2O2}$\\\hline
		与水反应&$\ce{Na2O + H2O -> 2NaOH}$&$\ce{2Na2O2 + 2H2O -> 4NaOH + O2 ^}$\\\hline
		与酸反应&$\ce{Na2O + 2H+ -> 2Na+ + H2O}$&$\ce{2Na2O2 + 4H+ -> 4Na+ + 2H2O + O2 ^}$\\\hline
		与\ce{CO2}反应&$\ce{Na2O + CO2 -> Na2CO3}$&$\ce{2Na2O2 + 2CO2 -> 2Na2CO3 + O2}$\\\hline
		用途&制取烧碱&漂白剂、消毒剂、供氧剂\\\hline
	\end{tabular}
	\end{center}
	\paragraph{过氧化钠的强氧化性}
	\begin{itemize}
		\item 与\ce{SO2}反应:$\ce{Na2O2 + SO2 -> Na2SO4}$
		\item 投入\ce{FeCl2}溶液中生成\ce{Fe(OH)3}沉淀
		\item 投入氢硫酸,氧化硫化氢成硫单质,溶液浑浊
		\item 氧化\ce{SO3^2-}成\ce{SO4^2-}
		\item 使品红溶液褪色
	\end{itemize}
	
	\subsubsection{碳酸钠和碳酸氢钠}
	\paragraph{碳酸钠\ce{Na2CO3}}
	\begin{itemize}
		\item 俗名:纯碱、苏打
		\item 与盐酸反应:$\ce{Na2CO3 + 2HCl -> 2NaCl + H2O + CO2 ^}$
		\item 与\ce{Ca(OH)2}溶液反应:$\ce{Na2CO3 + Ca(OH)2 -> CaCO3 v + 2NaOH}$
		\item 与\ce{BaCl2}溶液反应:$\ce{Na2CO3 + BaCl2 -> BaCO3 v + 2NaCl}$
	\end{itemize}
	\paragraph{碳酸氢钠\ce{NaHCO3}}
	\begin{itemize}
		\item 俗名:小苏打
		\item 与盐酸反应:$\ce{NaHCO3 + HCl -> NaCl + H2O + CO2 ^}$
		\item 与过量\ce{Ca(OH)2}溶液反应:$\ce{Ca2+ + OH- + HCO3- -> CaCO3 v + H2O}$
		\item 与少量\ce{Ca(OH)2}溶液反应:$\ce{Ca2+ + 2OH- + 2HCO3- + Ca(OH)2 -> CaCO3 v + 2H2O + CO3^2-}$
		\item 与\ce{BaCl2}溶液反应:无明显现象
		\item 受热分解:$\ce{2NaHCO3 ->[\Delta] Na2CO3 + H2O + CO2 ^}$
	\end{itemize}
	\paragraph{相互转换}
	$\ce{Na2CO3 <=>[CO2 + H2O{或}H+][\Delta({固体}){或}OH-] NaHCO3}$
	\paragraph{鉴别\ce{Na2CO3}和\ce{NaHCO3}}
	\subparagraph{固体}
	根据热稳定性加热,能产生使澄清石灰水变浑浊的气体的是\ce{NaHCO3}
	\subparagraph{溶液}
	\begin{itemize}
		\item 与可溶性钙、钡盐生成沉淀的是\ce{Na2CO3}
		\item 与足量盐酸反应剧烈的是\ce{NaHCO3}
		\item 逐滴加盐酸先生成气体的是\ce{NaHCO3}
		\item 等物质的量pH值较大的是\ce{Na2CO3}
	\end{itemize}
	
	
	\section{Mg和Al}
	
	\subsection{Mg单质和Al单质}
	\subsubsection{化学性质}
	\paragraph{与非金属单质反应}
	\begin{itemize}
		\item 与\ce{O2}反应:$\left\{\begin{array}{lr}
					\ce{2Mg + O2 ->[{点燃}] 2MgO}(耀眼白光)\\
					\ce{4Al + 3O2 ->[{点燃}] 2Al2O3}\\
				\end{array}\right.$
		\item 与\ce{CO2}反应:$\ce{2Mg + CO2 ->[{点燃}] 2MgO + C}(耀眼白光,黑色固体生成)$
		\item 与\ce{N2}反应:$\ce{3Mg + N2 ->[{点燃}] Mg3N2}$
		\item 与卤素反应:$\left\{\begin{array}{lr}
					\ce{2Mg + Cl2 ->[{点燃}] 2MgCl2}\\
					\ce{2Al + 3Cl2 ->[{点燃}] 2AlCl3}\\
				\end{array}\right.$
		\item 与硫反应:$\left\{\begin{array}{lr}
					\ce{Mg + S ->[\Delta] MgS}\\
					\ce{2Al + 3S ->[\Delta] Al2S3}\\
				\end{array}\right.$
	\end{itemize}
	注意,镁在空气中燃烧时会同时发生前三个反应。
	\paragraph{与热水反应}
	$\left\{\begin{array}{lr}
		\ce{Mg + H2O({沸水}) -> Mg(OH)2 + H2 ^}\\
		\ce{2Al + 6H2O -> 2Al(OH)3 + 3H2 ^}\\
	\end{array}\right.$
	\paragraph{与酸发生置换反应}
	特例:铝在冷的浓硫酸或浓硝酸中钝化。
	\paragraph{铝热反应}
	可以与\ce{FeO}、\ce{Fe2O3}、\ce{Fe3O4}、\ce{Cr2O3}、\ce{MnO2}、\ce{V2O5}等氧化物反应。\\
	$\left\{\begin{array}{lr}
		\ce{2Al + Fe2O3 ->[{高温}] Al2O3 + 2Fe}\\
		\ce{2Al + Cr2O3 ->[{高温}] Al2O3 + 2Cr}\\
	\end{array}\right.$
	\\用途:焊接金属、冶炼难溶金属。
	\paragraph{与碱反应}
	镁不与碱反应。铝与强碱发生反应:$\ce{2Al + 2NaOH + 6H2 -> 2NaAlO2 + 4H2O + 3H2 ^}$
	\subsubsection{制备}
	\begin{itemize}
		\item 工业制铝:$\ce{2Al2O3(l) ->[{冰晶石}][{通电}] 4Al + 3O2 ^}$
		\item 工业制镁:$\left\{\begin{array}{lr}
						\ce{Mg2+ + 2OH- -> Mg(OH)2 v}\\
						\ce{Mg(OH)2 + 2HCl -> MgCl2 + H2O}\\
						\ce{MgCl2(l) ->[{通电}] Mg + Cl2 ^}\\
					\end{array}\right.$
	\end{itemize}
	
	
	\subsection{铝、氧化铝和氢氧化铝的两性}
	\paragraph{与酸反应}
	$\left\{\begin{array}{lr}
		\ce{2Al + 6H+ -> 2Al3+ + 3H2 ^}({非氧化性酸})\\
		\ce{Al2O3 + 6H+ -> 2Al3+ + 3H2O}\\
		\ce{Al(OH)3 + 3H+ -> Al3+ + 3H2O}\\
	\end{array}\right.$
	\paragraph{与强碱反应}
	$\left\{\begin{array}{lr}
		\ce{2Al + 2OH- + 2H2O -> 2AlO2- + 3H2 ^}\\
		\ce{Al2O3 + 2OH- -> 2AlO2- + H2O}\\
		\ce{Al(OH)3 + OH- -> AlO2- + 2H2O}\\
	\end{array}\right.$
	\paragraph{\ce{Al(OH)3}的电离}
	$\left\{\begin{array}{lr}
		\ce{Al(OH)3 <=> H+ + AlO2- + H2O}\\
		\ce{Al(OH)3 <=> Al3+ + OH-}\\
	\end{array}\right.$
	
	\subsection{铝离子和偏铝酸根}
	\subsubsection{铝离子}
	\paragraph{与\ce{NaOH}的相互滴加}
	缓慢滴加并搅拌
	\subparagraph{将\ce{NaOH}滴入\ce{Al3+}溶液中}
	\begin{enumerate}
		\item 先出现白色沉淀:$\ce{Al3+ + 3OH- -> Al(OH)3 v}\\$
		\item 后沉淀消失:$\ce{Al(OH)3 + OH- -> AlO2- + 2H2O}\\$
	\end{enumerate}
	\subparagraph{将\ce{Al3+}滴入\ce{NaOH}溶液中}
	\begin{enumerate}
		\item 先无明显现象:$\ce{Al3+ + 4OH- -> AlO2- + H2O}\\$
		\item 后产生白色沉淀:$\ce{Al3+ + 3AlO2- + 6H2O -> 4Al3(OH)3 v}\\$
	\end{enumerate}
	\paragraph{与氨水反应}
	$\ce{Al3+ + NH3*H2O -> Al(OH)3 v + 3NH4+}\\$
	\subsubsection{偏铝酸根}
	\paragraph{与强酸相互滴加}缓慢滴加并搅拌
	\subparagraph{将\ce{H2SO4}滴入\ce{AlO2-}溶液中}
	\begin{enumerate}
		\item 先出现白色沉淀:$\ce{AlO2- + H+ + H2O -> Al(OH)3 v}\\$
		\item 后沉淀消失:$\ce{Al(OH)3 + 3H+ -> Al3+ + 3H2O}\\$
	\end{enumerate}
	\subparagraph{将\ce{AlO2-}滴入\ce{H2SO4}溶液中}
	\begin{enumerate}
		\item 先无明显现象:$\ce{AlO2- + 4H+ -> Al3+ + 2H2O}\\$
		\item 后产生白色沉淀:$\ce{Al3+ + 3AlO2- + 6H2O -> 4Al3(OH)3 v}\\$
	\end{enumerate}
	\paragraph{与碳酸反应}
	立即生成\ce{Al(OH)3}沉淀且不溶解。
	\begin{itemize}
		\item \ce{CO2}过量:$\ce{AlO2- + 2H2O + CO2 -> Al(OH)3 v + HCO3-}\\$
		\item \ce{CO2}少量:$\ce{2AlO2- + 3H2O + CO2 -> 2Al(OH)3 v + CO3^2-}\\$
	\end{itemize}
	\paragraph{与铵盐溶液反应}
	$\ce{NH4+ + AlO2- + H2O -> Al(OH)3 v + NH3 ^}\\$
	
	\subsection{氢氧化铝}
	\subsubsection{制备}
	\begin{itemize}
		\item $\ce{Al3+ + NH3*H2O -> Al(OH)3 v + 3NH4+}\\$
		\item $\ce{AlO2- + 2H2O + CO2 -> Al(OH)3 v + HCO3-}\\$
		\item $\ce{Al3+ + 3AlO2- + 6H2O -> 4Al3(OH)3 v}\\$
	\end{itemize}
\end{document}
